% -*- root: main.tex -*-

%*******************************************************************************
% Definicje stylu dokumentu
%*******************************************************************************

%===============================================================================
% klasa dokumentu

\documentclass[12pt, a4paper, oneside, titlepage, final]{book}

%===============================================================================
% Pakiety
\usepackage[MeX]{polski}
\usepackage[utf8]{inputenc}
\usepackage{anysize}
\usepackage{float}
\usepackage{subfigure}
\usepackage{multirow}
\usepackage{textcomp}
\usepackage{amsmath}
\usepackage{amssymb}
\usepackage{pgfplots}
\usepackage{tikz, calc}
\usetikzlibrary{arrows,shadows}
\pgfplotsset{compat=1.8}
\usepackage{verbatimbox}
\usepackage{hhline}
\usepackage{tikz-uml}
\usepackage{hyperref}
\usepackage{fancyhdr}

\linespread{1.3}								% 1.3 do interlinii 1.5

%===============================================================================
% Ustawienia dokumentu

\frenchspacing

% ustawienia wymiar�w
\oddsidemargin 20mm							% margines nieparzystych stron
\evensidemargin 20mm							% margines parzystych stron
\headheight 15pt								% wysoko�� paginy g�rnej
\topmargin 0mm									% margines g�rny

% styl paginacji
\pagestyle{fancy}
\renewcommand{\chaptermark}[1]{\markboth{#1}{}}
\renewcommand{\sectionmark}[1]{\markright{\thesection\ #1}{}}

% nag��wek 
\fancyhf{}
\fancyhead[L,RO]{\thepage}
\fancyhead[LO]{\small\nouppercase{\rightmark}}
%\fancyhead[R]{\small\nouppercase{\leftmark}}
\renewcommand{\headrulewidth}{0.1pt}
\renewcommand{\footrulewidth}{0pt}

% nag��wek w stylu plain 
\fancypagestyle{plain}
{
\fancyhf{}
\renewcommand{\headrulewidth}{0pt}
\renewcommand{\footrulewidth}{0pt}
}

% ta sekwencja tworzy czyste kartki na stronach po \cleardoublepage
\makeatletter
\def\cleardoublepage{\clearpage\if@twoside \ifodd\c@page\else
	\hbox{}
	\vspace*{\fill}
	\thispagestyle{empty}
	\newpage
	\if@twocolumn\hbox{}\newpage\fi\fi\fi}
\makeatother

%===============================================================================
% Zmienne �rodowiskowe i polecenia

% definicja
\newtheorem{definition}{Definicja}[chapter]

% twierdzenie
\newtheorem{theorem}{Twierdzenie}[chapter]

% obcoj�zyczne nazwy
\newcommand{\foreign}[1]{\emph{#1}}

% pozioma linia
\newcommand{\horline}{\noindent\rule{\textwidth}{0.4mm}}

% wstawianie obrazk�w {plik}{caption}{opis}
\newcommand{\fig}[3]
{
\begin{figure}[!htb]
\begin{center}
\includegraphics[width=\textwidth]{#1}
\caption[#2]{#2. #3}
\label{#1}
\end{center}
\end{figure}
}

%===============================================================================
