% -*- root: main.tex -*-

% title:
\begin{titlepage}

 \begin{tabular}{ll}
  \multirow{3}{*}{\includegraphics[scale=0.3]{figures/pw.jpg}} & POLITECHNIKA WARSZAWSKA                      \\
                                                               & Wydział Elektroniki i~Technik Informacyjnych \\
                                                               & Instytut Informatyki
 \end{tabular}
 
 \begin{flushright}
  Rok akademicki 2013/2014
 \end{flushright}

 \vspace{1.5cm}
 
 \begin{center}
  \LARGE PRACA DYPLOMOWA MAGISTERSKA
  
  \vspace{1.5cm}
  
  \large Jakub Turek
  
  \vspace{1.5cm}
  
  \textbf{Mechanizm modelowania danych i~mapowania obiektowego dla Apache Cassandry}
 \end{center}
 
 \vspace{1.5cm}
 
 \hfill Praca wykonana pod kierunkiem
 
 \hfill dra inż. Jakuba Koperwasa
 
 \vspace{1.5cm}

 \begin{flushleft}
  \begin{minipage}{7cm}
   Ocena: \dotfill \\ \\
   \hspace*{0cm} \dotfill \\[-0.7cm]
   \begin{center}
    \small\textit{Podpis Przewodniczącego Komisji Egzaminu Dyplomowego}
   \end{center}
  \end{minipage}
 \end{flushleft}

\end{titlepage}


% biography:
\newpage
\thispagestyle{empty}

\begin{flushright}
  Kierunek: Informatyka \\
  Specjalność: Inżynieria Systemów Informatycznych \\
  Data urodzenia: 1990.01.09 \\
  Data rozpoczęcia studiów: 2013.02.20 \\
\end{flushright}

\vspace{0.5cm}

\begin{center}
  \textbf{\textbf{Życiorys}}
\end{center}

\vspace{0.5cm}
 
Urodziłem się 9 stycznia 1990 roku w~Łodzi. W 1997 roku rozpocząłem edukację w~Szkole Podstawowej nr 7 w~Łodzi. W latach 2003-2006 kontynuowałem naukę w~Gimnazjum nr 42 im. Władysława Stanisława Reymonta w~Łodzi. Od 2006 roku uczyłem się w~Liceum Ogólnokształcącym nr 31 im. Ludwika Zamenhofa w~Łodzi. W~2009 roku zdałem egzaminy maturalne i~ukończyłem szkołę licealną z~wyróżnieniem. W~latach 2009-2013 studiowałem dziennie informatykę na Wydziale Elektroniki i~Technik Informacyjnych Politechniki Warszawskiej. Ukończyłem studia z~wynikiem celującym i~odebrałem tytuł zawodowy inżyniera. Obecnie kończę pracę dyplomową magisterską pod kierownictwem Instytutu Informatyki. We wrześniu 2012 roku rozpocząłem pracę zawodową jako programista aplikacji do zarządzania procesami biznesowymi oraz aplikacji mobilnych w~firmie Xentivo, gdzie pracuję do dziś. Moją pasją jest tworzenie aplikacji mobilnych oraz internetowych, które uruchamiane są w~środowisku iOS.
 
\vspace{0.5cm}

\begin{flushright}
  \begin{minipage}{5cm}
   \dotfill \\[-0.7cm]
   \begin{center}
   \small Podpis studenta
   \end{center}
  \end{minipage}
 \end{flushright}

\vspace{0.5cm}
 
 \begin{flushleft}
  Egzamin dyplomowy: \\
  Złożył egzamin dyplomowy w dniu: \dotfill \\
  z wynikiem: \dotfill \\
  Ogólny wynik studiów: \dotfill \\
  Dodatkowe uwagi i~wnioski Komisji: \dotfill \\
  \hspace{0cm} \dotfill
\end{flushleft}
 
  
% % abstract:
\newpage
\thispagestyle{empty}
 
\textbf{Streszczenie} \\
 
  Celem pracy dyplomowej jest stworzenie mechanizmu do modelowania i~mapowania obiektowego dla aplikacji wykorzystujących bazę danych Apache Cassandra. System powinien cechować się wysoką wydajnością i~wsparciem dla wzorców modelowania. Dodatkowym celem jest zachowanie zgodności z~istniejącymi mechanizmami mapowania dla baz relacyjnych. Nie wszystkie cele udało się zrealizować. Stworzenie wydajnego mapowania wymagało odejścia od relacyjnego modelu danych. Praca dyplomowa zawiera zbiór informacji teoretycznych, badań oraz studia przypadków, które posłużyły do nakreślenia pryncypiów systemu modelowania obiektowego. Obejmuje również implementację omawianych mechanizmów.
 
 \vspace*{\stretch{1}}
 
 \begin{center}
  \large \textbf{Data modeling and object mapping mechanism for Apache Cassandra}
 \end{center}

 \vspace*{1cm}
 
 \textbf{Summary} \\
 
  The goal of this thesis is to create a~modeling and object mapping mechanism for applications which use Apache Cassandra database. The mechanism should provide good performance and be capable of making use of Cassandra design patterns. An additional goal is to explore possibility of using existing interfaces for object-relational mapping. Not all goals could be achieved. Creating a~fast system required abandoning object-relational mapping compatibility. The thesis contains of theoretical information, research results and case studies which contributed to specifying principles of the mechanism. It also provides functional implementation of described ideas.
 
\vspace*{\stretch{1}}