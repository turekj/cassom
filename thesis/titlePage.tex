% title:
\begin{titlepage}

 \begin{tabular}{ll}
  \multirow{3}{*}{\includegraphics[scale=0.3]{figures/pw.jpg}} & POLITECHNIKA WARSZAWSKA                      \\
                                                               & Wydział Elektroniki i~Technik Informacyjnych \\
                                                               & Instytut Informatyki
 \end{tabular}
 
 \begin{flushright}
  Rok akademicki 2013/2014
 \end{flushright}

 \vspace{2cm}
 
 \begin{center}
  \LARGE PRACA DYPLOMOWA MAGISTERSKA
  
  \vspace{2cm}
  
  \large Jakub Turek
  
  \vspace{2cm}
  
  \textbf{[TYTUŁ]}
 \end{center}
 
 \vspace{3cm}
 
 \hfill Praca wykonana pod kierunkiem
 
 \hfill dra inż. Jakuba Koperwasa
 
 \vspace{3cm}

 \begin{flushleft}
  \begin{minipage}{7cm}
   Ocena: \dotfill \\ \\
   \hspace*{0cm} \dotfill \\[-0.7cm]
   \begin{center}
    \small\textit{Podpis Przewodniczącego Komisji Egzaminu Dyplomowego}
   \end{center}
  \end{minipage}
 \end{flushleft}

\end{titlepage}


% biography:
\newpage
\thispagestyle{empty}

\begin{flushright}
  Kierunek: Informatyka \\
  Specjalność: Inżynieria Systemów Informatycznych \\
  Data urodzenia: 1990.01.09 \\
  Data rozpoczęcia studiów: 2013.02.20 \\
\end{flushright}

\vspace*{3cm}

\begin{center}
  \textbf{\textbf{Życiorys}}
\end{center}
 
\vspace{1cm}

Urodziłem się 9 stycznia 1990 roku w~Łodzi. W 1997 roku rozpocząłem edukację w~Szkole Podstawowej nr 7 w~Łodzi. W latach 2003-2006 kontynuowałem naukę w~Gimnazjum nr 42 im. Władysława Stanisława Reymonta w~Łodzi. Od 2006 roku uczyłem się w~Liceum Ogólnokształcącym nr 31 im. Ludwika Zamenhofa w~Łodzi. W~2009 roku zdałem egzaminy maturalne i ukończyłem szkołę licealną z wyróżnieniem. W~latach 2009-2013 studiowałem dziennie informatykę na Wydziale Elektroniki i~Technik Informacyjnych Politechniki Warszawskiej. Ukończyłem studia z~wynikiem celującym i~odebrałem tytuł zawodowy inżyniera. Obecnie kończę Pracę Dyplomową Magisterską pod kierownictwem Instytutu Informatyki. We wrześniu 2012 roku rozpocząłem pracę zawodową jako programista aplikacji do zarządzania procesami biznesowymi oraz aplikacji mobilnych w~firmie Xentivo, gdzie pracuję do dziś. Moją pasją jest tworzenie aplikacji mobilnych oraz internetowych, które uruchamiane są w~środowisku iOS.

\vspace{2cm}
 
\begin{flushright}
  \begin{minipage}{5cm}
   \dotfill \\[-0.7cm]
   \begin{center}
   \small Podpis studenta
   \end{center}
  \end{minipage}
 \end{flushright}
 
 \vspace{3cm}
 
 \begin{flushleft}
  Egzamin dyplomowy: \\
  Złożył egzamin dyplomowy w dniu: \dotfill \\
  z wynikiem: \dotfill \\
  Ogólny wynik studiów: \dotfill \\
  Dodatkowe uwagi i~wnioski Komisji: \dotfill \\
  \hspace{0cm} \dotfill
\end{flushleft}
 
  
% % abstract:
\newpage
\thispagestyle{empty}
 
\textbf{Streszczenie} \\
 
  Celem Pracy Dyplomowej jest stworzenie interfejsu programowania aplikacji umożliwiającego efektywne rozwiązywanie problemów w~modelowaniu dziedziny danych opartych o~bazę danych Apache Cassandra.
 
 \vspace*{\stretch{1}}
 
 \begin{center}
  \large \textbf{Client-server Augmented Reality applications framework for Android system}
 \end{center}

 \vspace*{1cm}
 
 \textbf{Summary} \\
 
  The goal of this thesis is to create a~framework supporting the development of multiuser applications for Android system.
  The framework should consist of a~client-server bus, as well as several Augmented Reality components.
  An additional goal is to explore the possibility of adapting commonly used programming practises dedicated to large projects during Andorid applications development.
  All the goals have been fully accoplished - the output is the implementation of the client-server bus as well as AR components allowing to track the device's geographic coordinates and to render a~stable three-dimensional graphics on the display of the device.
  In order to demonstrate the features of the framework, a~sample multiplayer game has been created.
  The thesis includes a~description of the created components' design process, as well as their functionality and structure.
 
\vspace*{\stretch{1}}