% -*- root: main.tex -*-

\chapter{Modelowanie obiektowe dla Cassandry}

Na podstawie wyników wydajnościowych przedstawionych w~sekcji~\ref{sec:cassandra_orm_performance_summary} Autor zaproponował odmienną koncepcję mechanizmu mapowania obiektowego dla bazy danych Cassandra. Aby uwypuklić różnice pomiędzy przedstawioną propozycją i~tradycyjnymi mechanizmami ORM Autor postuluje stosowanie odmiennego nazewnictwa. Omówiona w~sekcji~\ref{sec:om_for_cassandra_concept} koncepcja, która będzie rozwijana w~dalszej części pracy, nazywana będzie \textbf{modelowaniem obiektowym}:

\begin{itemize}
	\item Mechanizmy mapowania obiektowo-relacyjnego nie wymuszają na użytkowniku kolejności projektowania komponentów. Osiągalne są zarówno podejście \emph{code first}\footnote{Code first (ang. dosłownie ,,najpierw kod źródłowy'') - w~kontekście ORM oznacza to modelowanie dziedziny za pomocą kodu źródłowego, z~którego następnie generowany jest schemat bazy danych.}, jak i~\emph{database first}\footnote{Database first (ang. dosłownie ,,najpierw baza danych'') - w~kontekście ORM oznacza to modelowanie dziedziny jako schemat bazodanowy, do którego następnie pisany jest mapujący kod źródłowy.}. W~proponowanym przez Autora modelowaniu obiektowym, ze względu na występowanie struktur wysokiego poziomu odpowiadających wzorcom projektowym, podejście \emph{database first} jest możliwe tylko teoretycznie lub w~bardzo ograniczonym zakresie. Znacznie bardziej efektywne jest podejście \emph{code first}. Posiada ono wsparcie implementacyjne - generację schematu - i~jest rekomendowane przez Autora.
	\item W~mechanizmach mapowania obiektowo-relacyjnego mapowanie odpowiada najczęściej typom danych, natomiast w~obrębie danego typu jest ono jednoznaczne. Przykładowo identyfikator konwertowany jest na liczbowy klucz główny, lista przechowywana jest jako relacja, a~ciąg znaków przekształcany jest na typ \emph{VARCHAR}. W~przypadku modelowania obiektowego dla Cassandry wybór wyznaczać będzie natomiast sposób przechowywania danej wartości. W~zależności od narzuconego sposobu modelowania listy będzie mogła zostać ona zrzutowana na wartość kolumny o~typie \verb+list+, nazwy kolumn w~wierszu lub osobną tabelę.
	\item Mapowanie obiektowo-relacyjne dostarcza przede wszystkim niskopoziomowe odpowiedniki typów bazodanowych. Wyjątkiem jest relacja wiele-do-wielu. Modelowanie obiektowe dla Cassandry skupia się zarówno na polach niskiego poziomu (przykładowo kolumna o~wartości tekstowej), jak i~na reużywalności komponentów wysokopoziomowych odpowiadających wzorcom projektowym.
	\item Częścią modelowania obiektowego dla Cassandry są opcjonalne mechanizmy mapowania obiektowo-relacyjnego. Należą do nich generacja schematu oraz migracje pomiędzy wersjami modelu.
\end{itemize}