% -*- root: main.tex -*-

\chapter{Modelowanie obiektowe dla Cassandry}

Na podstawie wyników wydajnościowych przedstawionych w~sekcji~\ref{sec:cassandra_orm_performance_summary} Autor zaproponował odmienną koncepcję mechanizmu mapowania obiektowego dla bazy danych Cassandra. Aby uwypuklić różnice pomiędzy przedstawioną propozycją i~tradycyjnymi mechanizmami ORM Autor postuluje stosowanie odmiennego nazewnictwa. Omówiona w~sekcji~\ref{sec:om_for_cassandra_concept} koncepcja, która będzie rozwijana w~dalszej części pracy, nazywana będzie \textbf{modelowaniem obiektowym}:

\begin{itemize}
	\item Mechanizmy mapowania obiektowo-relacyjnego nie wymuszają na użytkowniku kolejności projektowania komponentów. Osiągalne są zarówno podejście \emph{code first}\footnote{Code first (ang. dosłownie ,,najpierw kod źródłowy'') - w~kontekście ORM oznacza to modelowanie dziedziny za pomocą kodu źródłowego, z~którego następnie generowany jest schemat bazy danych.}, jak i~\emph{database first}\footnote{Database first (ang. dosłownie ,,najpierw baza danych'') - w~kontekście ORM oznacza to modelowanie dziedziny jako schemat bazodanowy, do którego następnie pisany jest mapujący kod źródłowy.}. W~proponowanym przez Autora modelowaniu obiektowym, ze względu na występowanie struktur wysokiego poziomu odpowiadających wzorcom projektowym, podejście \emph{database first} jest możliwe tylko teoretycznie lub w~bardzo ograniczonym zakresie. Znacznie bardziej efektywne jest podejście \emph{code first}. Posiada ono wsparcie implementacyjne - generację schematu - i~jest rekomendowane przez Autora.
	\item W~mechanizmach mapowania obiektowo-relacyjnego mapowanie odpowiada najczęściej typom danych, natomiast w~obrębie danego typu jest ono jednoznaczne. Przykładowo identyfikator konwertowany jest na liczbowy klucz główny, lista przechowywana jest jako relacja, a~ciąg znaków przekształcany jest na typ \emph{VARCHAR}. W~przypadku modelowania obiektowego dla Cassandry wybór wyznaczać będzie natomiast sposób przechowywania danej wartości. W~zależności od narzuconego sposobu modelowania listy będzie mogła zostać ona zrzutowana na wartość kolumny o~typie \verb+list+, nazwy kolumn w~wierszu lub osobną tabelę.
	\item Mapowanie obiektowo-relacyjne dostarcza przede wszystkim niskopoziomowe odpowiedniki typów bazodanowych. Wyjątkiem jest relacja wiele-do-wielu. Modelowanie obiektowe dla Cassandry skupia się zarówno na polach niskiego poziomu (przykładowo kolumna o~wartości tekstowej), jak i~na reużywalności komponentów wysokopoziomowych odpowiadających wzorcom projektowym.
	\item Częścią modelowania obiektowego dla Cassandry są opcjonalne mechanizmy mapowania obiektowo-relacyjnego. Należą do nich generacja schematu oraz migracje pomiędzy wersjami modelu.
\end{itemize}

W~dalszej części pracy omówiona zostanie konkretna implementacja modelowania obiektowego dostarczona wykonana przez Autora. Pełna nazwa tego projektu to \textbf{Object Modeling for Cassandra}. W~dalszej części pracy używana będzie nazwa skrócona - \emph{OMC}.

\section{OMC - podstawowe pojęcia}

Środowisko aplikacyjne OMC definiuje trzy podstawowe pojęcia: model, pole oraz silnik.

\subsection{Model}

Jest to klasa, która opisuje atomowy względem kodu źródłowego obiekt modelujący bazę danych. Obiekt ten, w~zależności od definicji, może być mapowany na jedną lub wiele tabel. Jedynym wymaganiem, które musi spełnić klasa należąca do modelu jest dziedziczenie po klasie bazowej \verb+Model+. Ciało modelu stanowią pola.

\subsection{Pole}

Pole jest to zmienna klasy definiującej model. Opisuje atomową wartość z~punktu widzenia kodu źródłowego. W~zależności od typu pole może być mapowane na jedną kolumnę, wiele kolumn lub autonomiczną tabelę w~bazie danych. Aby pole klasy mogło zostać zakwalifikowane jako pole modelu, w~definicji klasy należy mu przypisać obiekt dziedziczący po klasie bazowej \verb+Field+.

\subsection{Silnik}

Silnik jest to obiekt, który odpowiada za nawiązywanie połączenia z~bazą danych. Po utworzeniu silnika wystarczy przypisać go do całego modelu. Wszystkie operacje bazodanowe będą domyślnie wykonywane przez przypisany silnik. 

Utworzenie silnika wymaga podania adresu URL opisującego połączenie, nazwanego z~języka angielskiego \emph{connection string}. Strukturę tego adresu przedstawia listing~\ref{lst:connection_string}.

\begin{verbbox}
	cassandra://{ip}:{port}/{keyspace}?rf={rf}&strategy={strategy}
\end{verbbox}

\begin{figure}[ht!]
	\centering
	\theverbbox
	\caption{Struktura adresu URL opisującego połączenie z~bazą danych Cassandra.}
	\label{lst:connection_string}
\end{figure}

Wszystkie elementy ujęte w~znaki \textbf{\{\}} w~adresie~\ref{lst:connection_string} należy zamienić na wartości według poniższego objaśnienia:

\begin{description}
	\item[ip] adres IP węzła koordynatora, z~którym będzie komunikować się kod źródłowy,
	\item[port] port IP koordynatora (podanie portu jest opcjonalne; wartość domyślna to 9042),
	\item[keyspace] przestrzeń nazw Cassandry, w~której operuje OCM,
	\item[rf] współczynnik replikacji\footnote{Współczynnik replikacji (ang. replication factor) - określa na ile węzłów kopiowany jest każdy wstawiony wiersz.} podawany przy tworzeniu przestrzeni nazw, jeżeli ta nie istnieje (parametr jest opcjonalny),
	\item[strategy] nazwa strategii replikacji podawanej przy tworzeniu przestrzeni nazw, jeśli ta nie istnieje (parametr jest opcjonalny).
\end{description}

Do utworzenia silnika służy metoda \verb+Engine.create_engine(connection_string)+. Aby przypisać silnik do modelu należy użyć metody \verb+Model.bind(engine)+. Przykład wywołania przypisującego modelowi silnik dla serwera lokalnego i~przestrzeni nazw \verb+test+ został zaprezentowany na listingu~\ref{lst:engine_creation}.

\begin{verbbox}
	engine = Engine.create_engine('cassandra://127.0.0.1/test')
	Model.bind(engine)
\end{verbbox}

\begin{figure}[ht!]
	\centering
	\theverbbox
	\caption{Przykład wywołania tworzącego silnik i~przypisującego go do modelu.}
	\label{lst:engine_creation}
\end{figure} 

\section{Lista życzeń użytkownika - studium przypadku}



\begin{verbbox}
class User(Model):
    id = TimeUuidField(autotimestamp=True, partition_key=True)
    name = TextField(selectable=True)
    surname = TextField(selectable=True)
\end{verbbox}

\begin{figure}[ht!]
	\centering
	\theverbbox
	\caption{Tabela użytkownika opisana przy pomocy OMC.}
\end{figure}

\begin{verbbox}
class Item(Model):
    id = TimeUuidField(autotimestamp=True, partition_key=True)
    name = TextField(length=255)
    price = NumberField(decimal_positions=2)
    desc = TextField()
    category = DictionaryField(entries=['BOOKS', 'CLOTHES', 'FURNITURE'])
    weight = NumberField(decimal_positions=1)
\end{verbbox}

\begin{figure}[ht!]
	\centering
	\theverbbox
	\caption{Tabela przedmiotu opisana przy pomocy OMC.}
\end{figure}

\begin{verbbox}
class Wishlist(Model):
    userId = TextField(partition_key=True)
    item = DenormalizedField(relates=Item, 
                             clustering_keys=['itemId'], 
                             fields=['name', 'price'])
\end{verbbox}

\begin{figure}[ht!]
	\centering
	\theverbbox
	\caption{Tabela listy życzeń opisana przy pomocy OMC.}
\end{figure}
