% -*- root: main.tex -*-

\chapter{Wstęp}

Tematem pracy jest zbadanie możliwości stworzenia systemu mapowania obiektowego dla bazy danych Apache Cassandra. Celem pracy jest zweryfikowanie możliwości wykorzystania interfejsów analogicznych mechanizmów stworzonych dla relacyjnych baz danych. Podstawowym wymaganiem jest zachowanie wysokiej wydajności zapisu, która wyróżnia Cassandrę na tle innych systemów bazodanowych, a~także możliwość czerpania z~praktyk zwiększających efektywność modelowania opisanych przez użytkowników Cassandry. Niniejsza praca prezentuje zbiór rozważań teoretycznych, badań oraz doświadczeń z~użytkowania, które przyczyniły się do powstania takiego mechanizmu. W~skład pracy wchodzi także implementacja mechanizmu.

\section{Zakres pracy}

Zakres pracy obejmuje następujące elementy:

\begin{enumerate}
	\item Badanie możliwości wykorzystania interfejsów mapowania obiektowo-relacyjnego dla bazy danych Cassandra:
		\begin{itemize}
			\item Wyszukanie istniejących implementacji.
			\item Pomiary wydajności ze szczególnym naciskiem na różnice w~stosunku do relacyjnych baz danych oraz modelowania i~komunikacji z~wykorzystaniem języka zapytań Apache Cassandry.
		\end{itemize}
	\item Zaproponowanie własnego interfejsu zorientowanego na wysoką wydajność oraz wsparcie w~modelowaniu danych z~wykorzystaniem najlepszych praktyk:
		\begin{itemize}
			\item Zebranie i~opisanie sposobów modelowania zależności między danymi oraz porównanie ich wydajności w~różnych przypadkach użycia.
			\item Zebranie i~opisanie zbioru najlepszych wzorców modelowania oraz możliwości wspierania ich w~mechanizmie.
			\item Wykonanie referencyjnej implementacji zaproponowanego interfejsu.
			\item Przeprowadzenie badań wydajnościowych implementacji.
		\end{itemize}
\end{enumerate}

\section{Motywacja}

W~dobie szybkiego postępu technologicznego i~intensywnego rozwoju informatyki, a~zwłaszcza powszechnego dostępu do sieci Internet oraz niskich kosztów składowania danych, zaczął rozwijać się trend nazywany przez specjalistów od marketingu \emph{big data}\footnote{Z angielskiego oznacza to dosłownie ,,wielkie dane''.}. \emph{Big data} to termin używany do określania dużych rozmiarów, różnorodnych i~często zmieniających się zbiorów danych.~\cite{big_data_definition} Efektywne przetwarzanie takich danych wymaga stosowania innowacyjnych, stale ulepszanych rozwiązań technologicznych. 

Powstanie ogromnych portali społecznościowych obsługiwanych przez dziesiątki tysięcy fizycznych maszyn spowodowało pojawienie się wyzwań, które wcześniej nie były znane w~informatyce. Awarie maszyn, dotychczas traktowane jako sytuacje wyjątkowe, stały się regułą. W~połączeniu z~wymaganiami na niskie czasy obsługi setek tysięcy jednoczesnych żądań doprowadziło to do osiągnięcia kresu możliwości wykorzystywanych od wielu lat relacyjnych baz danych. Aby móc sprostać tym warunkom zaczęły powstawać nowe silniki bazodanowe, które odchodziły od postulatów transakcyjności i~klasycznej, tabelarycznej reprezentacji danych. Jednym z~najlepszych rozwiązań do obsługi wysoce rozproszonych środowisk jest Apache Cassandra.

Negatywną stroną wprowadzenia nowych rozwiązań z~zakresu przechowywania danych było odcięcie się od wielu mechanizmów, które bardzo ułatwiały pracę z~aplikacjami. Przede wszystkim nie było możliwe użycie istniejących systemów mapowania obiektowo-relacyjnego. Ich wykorzystanie stało się na tyle powszechne, że w~dniu dzisiejszym mogą być uznawane za standard w~tworzeniu aplikacji bazodanowych.

Wraz z~rozwojem Cassandry, a~zwłaszcza wprowadzeniem języka zapytań o~składni przypominającej ten z~relacyjnych baz danych, pojawiła się możliwość ponownego wykorzystania istniejących interfejsów. Nawet jeżeli zachowanie wysokiej wydajności wymaga wprowadzenia pewnych modyfikacji, to warto to uczynić. Dzięki temu możliwe jest zapewnienie narzędzia do szybszego rozwoju aplikacji oraz zmniejszenie bariery wejścia związanem z~opanowaniem podstaw nowego systemu.

Autor ma nadzieję, że przeprowadzone przez niego badania przyczynią się do rozwoju prac nad efektywnym modelowaniem danych w~Apache Cassandra oraz pomogą wypełnić lukę w~oprogramowaniu wspomagającemu pracę z~nierelacyjnymi bazami danych. Ponadto Autor wyraża opinię, że rozwój dedykowanych systemów mapowania przyczyni się do szerszego wykorzystywania przedmiotowego systemu bazodanowego.

\section{Zawartość pracy}

Rozdział~\ref{chap:apache_cassandra} opisuje podstawy teoretyczne związane z~silnikiem Apache Cassandra, które są niezbędne do zrozumienia całej pracy. W~sekcji~\ref{sec:cassandra_data_model} Autor przedstawia stosowany w~silniku model danych. Sekcja~\ref{sec:cassandra_data_distribution} opisuje sposób dystrybucji wierszy w~klastrze. W~punkcie~\ref{sec:relative_vs_cassandra_model} zostały omówione, na przykładzie, różnice pomiędzy relacyjnym modelem danych a~schematem Cassandry. Sekcja~\ref{sec:about_cql} opowiada o~języku zapytań CQL, natomiast w~punkcie~\ref{sec:patterns_and_antipatterns} zostało wprowadzone pojęcie wzorców i~antywzorców modelowania.

W~rozdziale~\ref{chap:object_relational_mapping} zostały przedstawione informacje teoretyczne na temat mapowania obiektowo-relacyjnego. W~punkcie~\ref{sec:jpa} omówione są zasady działania mechanizmu na przykładzie \emph{persistence API} języka Java. Kolejna sekcja (\ref{sec:kundera}) przedstawia wyniki badania wydajności istniejącej implementacji JPA dla Apache Cassandry. Na podstawie uzyskanych wyników Autor formuuje wnioski zaprezentowane w~punkcie~\ref{sec:cassandra_orm_performance_summary}, które są przyczynkiem do przedstawienia autorskiej koncepcji mapowania opisanej w~punkcie~\ref{sec:om_for_cassandra_concept}.

Rozdział~\ref{chap:cassandra_object_modeling} zawiera opis wszystkich unikalnych cech, które zawiera zaproponowane przez Autora narzędzie do modelowania dziedziny w~Cassandrze. W~sekcji~\ref{sec:ocm_library_description} wypisane są mechanizmy, dla których wsparcie posiada przygotowany interfejs. Punkt~\ref{sec:ocm_basics} wprowadza w~podstawowe pojęcia związane z~biblioteką. Sekcja~\ref{sec:ocm_model_definition} przedstawia sposób definiowania modelu. Punkt~\ref{sec:relation_modeling} to obszerne rozważania na temat modelowania zależności pomiędzy danymi, wraz z~kompleksowymi wynikami badań, które pomagają wybrać poprawny model w~zależności od scenariusza użycia. W~sekcji~\ref{sec:ocm_data_modeling_support} zaprezentowane zostały wszystkie wzorce projektowania schematu, dla których wsparcie udostępnia interfejs modelowania. Punkty~\ref{sec:batch_processing} oraz~\ref{sec:ocm_counters} specyfikują wsparcie biblioteki dla mechanizmów bazodanowych. W~sekcji~\ref{sec:ocm_migrations} Autor przedstawia mechanizm migracji danych dedykowany dla Apache Cassandry. Ostatni punkt rozdziału (\ref{sec:ocm_profiling}) przedstawia narzędzia do profilowania aplikacji udostępnione przez system modelowania obiektowego.