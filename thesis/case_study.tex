% -*- root: main.tex -*-

\chapter{Studium przypadku}
\label{chap:case_study}

W~rozdziale Autor przedstawia proces modelowania danych w~mechanizmie OMC. Przebieg procesu jest omówiony na przykładzie istniejących modeli danych, które są obsługiwane przy pomocy CQL. Dzięki takiemu podejściu łatwo jest wskazać korzyści, które zapewnia użytkownikowi stosowanie OMC.

\section{Twissandra}
\label{sec:case_study_twissandra}

W~sekcji \ref{sec:twissandra} został szczegółowo omówiony projekt Twissandra. Przedstawia on kompletny model danych w~języku CQL dla serwisu o~analogicznej funkcjonalności do platformy Twitter. 

\subsection{Użytkownik}

\begin{verbbox}
	class User(Model):
	    username = TextField(partition_key=True)
	    password = PasswordField(algorithm='SHA512', 
	                             iterations=10000)
\end{verbbox}

\begin{figure}[ht!]
	\centering
	\theverbbox
	\caption{Użytkownik Twissandry zamodelowany w~OMC.}
	\label{vrb:omc_twissandra_user}
\end{figure}

\begin{verbbox}
	class Followers(Model):
	    user = RelationField(relates=User, 
	                         partition_key=True)
	    follower = RelationField(relates=User,
	                             clustering_key=True,
	                             searchable=True)
	    since = TimestampField(auto_add_now=True)
\end{verbbox}

\begin{figure}[ht!]
	\centering
	\theverbbox
	\caption{Śledzeni użytkownicy w~Twissandrze zamodelowani w~OMC.}
	\label{vrb:omc_twissandra_followers}
\end{figure}

\begin{verbbox}
	class Tweet(Model):
	    id = UuidField(partition_key=True, 
	                   auto_generate=True)
	    user = RelationField(relates=User)
	    body = TextField()
\end{verbbox}

\begin{figure}[ht!]
	\centering
	\theverbbox
	\caption{Wpis w~Twissandrze zamodelowany w~OMC.}
	\label{vrb:omc_twissandra_tweet}
\end{figure}

\begin{verbbox}
	class Timeline(Model):
	    user = RelationField(relates=User,
	                         partition_key=True)
	    day = TextField()
	    time = UuidField(type=TimeUuid, 
	                     auto_generate=True,
	                     clustering_key=True)
	    tweet = DenormalizedField(relates=Tweet,
	                              fields=['id', 'body'])
\end{verbbox}

\begin{figure}[ht!]
	\centering
	\theverbbox
	\caption{caption}
	\label{vrb:label}
\end{figure}