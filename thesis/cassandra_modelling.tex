\chapter{Apache Cassandra}

Apache Cassandra jest bazą danych NoSQL\footnote{NoSQL (ang. Not Only SQL) - podzbiór baz danych, które zapewniają inne sposoby modelowania dziedziny niż tradycyjny model oparty na tabelach i~relacjach.}, która powstała w~wyniku połączenia rozwiązań wykorzystywanych w~Dynamo\footnote{Amazon DynamoDB - zdecentralizowana, wysoce skalowalna baza danych typu klucz-wartość.} oraz BigTable\footnote{Google BigTable - rozproszony system bazodanowy, który dobrze skaluje się dla ogromnych ilości danych. \cite{official_bigtable}}. Cassandra początkowo była rozwijana dla potrzeb portalu społecznościowego Facebook. Baza danych powstała z~myślą o~rozwiązaniu problemu pełnotekstowego przeszukiwania skrzynek odbiorczych użytkowników, w~których dziennie zapisywane były miliardy wiadomości. Głównym celem, do których dążyli twórcy Cassandry była możliwość wykorzystania jej do przechowywania ogromnych ilości danych w~bardzo rozproszonym środowisku, gdzie awarie pojedynczych węzłów zdarzają się na porządku dziennym. W~tych warunkach baza danych musi zapewniać szybki i~niezawodny dostęp do danych. \cite{cassandra_introduction} 

Apache Cassandra wykorzystywana jest w~wielu serwisach na całym świecie. Najbardziej znaczące przykłady użycia produkcyjnego to eBay\footnote{eBay - największy portal z~aukcjami internetowymi na świecie}, Instagram\footnote{Instagram - portal pozwalający na umieszczanie fotografii.} oraz GitHub\footnote{GitHub - usługa pozwalająca na przechowywanie i~wersjonowanie kodu źródłowego aplikacji.}. Największa światowa instalacja Cassandry obejmuje około 15000 węzłów, na których przechowywane jest łącznie ponad 4 petabajty danych. \cite{official_cassandra}

\section{Model danych}

