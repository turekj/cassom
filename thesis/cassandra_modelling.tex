% -*- root: main.tex -*-

\chapter{Apache Cassandra}

Apache Cassandra jest bazą danych NoSQL\footnote{NoSQL (ang. Not Only SQL) - podzbiór baz danych, które zapewniają inne sposoby modelowania dziedziny niż tradycyjny model oparty na tabelach i~relacjach.}, która powstała w~wyniku połączenia rozwiązań wykorzystywanych w~Dynamo\footnote{Amazon DynamoDB - zdecentralizowana, wysoce skalowalna baza danych typu klucz-wartość.} oraz BigTable\footnote{Google BigTable - rozproszony system bazodanowy, który dobrze skaluje się dla ogromnych ilości danych.}. Cassandra początkowo była rozwijana dla potrzeb portalu społecznościowego Facebook. Baza danych powstała z~myślą o~rozwiązaniu problemu pełnotekstowego przeszukiwania skrzynek odbiorczych użytkowników, w~których dziennie zapisywane były miliardy wiadomości. Głównym celem, do których dążyli twórcy Cassandry była możliwość wykorzystania jej do przechowywania ogromnych ilości danych w~bardzo rozproszonym środowisku, gdzie awarie pojedynczych węzłów zdarzają się na porządku dziennym. W~tych warunkach baza danych musi zapewniać szybki i~niezawodny dostęp do danych. \cite{cassandra_introduction} 

Apache Cassandra wykorzystywana jest w~wielu serwisach na całym świecie. Najbardziej znaczące przykłady użycia produkcyjnego to eBay\footnote{eBay - największy portal z~aukcjami internetowymi na świecie}, Instagram\footnote{Instagram - portal pozwalający na umieszczanie fotografii.} oraz GitHub\footnote{GitHub - usługa pozwalająca na przechowywanie i~wersjonowanie kodu źródłowego aplikacji.}. Największa światowa instalacja Cassandry obejmuje około 15000 węzłów, na których przechowywane jest łącznie ponad 4 petabajty danych. \cite{official_cassandra}

W~przeciwieństwie do relacyjnych baz danych, Apache Cassandra nie zapewnia wsparcia dla reguły ACID\footnote{ACID (ang. Atomic, Consistency, Isolation, Durability) - zasada atomowości, spójności, izolacji i~trwałości, które gwarantują poprawne przetwarzanie transakcji w~bazach danych.}. Zamiast tego zostały zrealizowane postulaty twierdzenia CAP: ,,we współdzielonym systemie plików można zachować maksymalnie dwie z~trzech właściwości: spójności, dostępności oraz podatności na partycjonowanie''. \cite{cap_theorem} Apache Cassandra priorytetyzuje właściwości dostępności oraz partycjonowania. Spójność danych jest odwrotnie proporcjonalna i~może być regulowana w~zależności od czasu odpowiedzi. Wysoka spójność danych oznacza wolniejszą odpowiedź bazy.

\section{Model danych}

Model danych Apache Cassandra jest analogiczny do BigTable. \cite{official_bigtable} Można przedstawić go jako dwuwymiarowa mapa trójek wartości:

\begin{figure}[ht!]
	\centering
	\verb+Map<RowKey, Map<ColumnKey, Triple<Value, Timestamp, TTL>>>+
\end{figure}

gdzie \verb+RowKey+ to identyfikator wiersza, \verb+ColumnKey+ to identyfikator kolumny, \verb+Value+ to wartość komórki, \verb+Timestamp+ to czas aktualizacji komórki, a~\verb+TTL+ to czas życia danej wartości. \cite{mc_fadin_long_live_data_model} Na rysunku \ref{fig:data_model_example} przedstawiona jest schematyczna ilustracja wiersza danych. Pogrubiona wartość w~lewej komórce to klucz wiersza, natomiast wyróżnione wartości w~pierwszym wierszu oznaczają klucze poszczególnych kolumn. Każda komórka składa się z~trzech wartości: wartości, czasu życia oraz ,,odcisku czasu''.

\begin{figure}[ht!]
	\centering
	\begin{tabular}{|l||c|c|c|c|}
		\hhline{|-||----|}
		& \textbf{ABC} & \textbf{DEF} & $\cdots$ & \textbf{XYZ} \\
		\hhline{|~||====|}
		\textbf{123} & test value & another test value & $\cdots$ & not a~test value \\
		\cline{2-5}
		\textbf{456} & $20$ & $\infty$ & $\cdots$ & $\infty$ \\
		\cline{2-5}
		& 1291987837942000 & 1291980736812000 & $\cdots$ & 1291980736212000 \\
		\hhline{|-||----|}
	\end{tabular}

	\caption{Przykładowy wiersz modelu danych o~identyfikatorze 123456. Wartość komórki (123456, DEF) to ,,another test value''.}
	\label{fig:data_model_example}
\end{figure}

\section{Dystrybucja danych}

Do dystrybucji danych wykorzystywana jest funkcja skrótu, która zachowuje kolejność elementów. Węzły są przedstawiane na w~topologii pierścienia. Algorytm dystrybucji zostanie omówiony na przykładzie ze schematu~\ref{fig:data_distribution}: 

\begin{figure}[ht!]
	\centering
	
	\begin{tikzpicture}
		\def \n {6}
		\def \radius {3cm}
		\def \margin {10}

		\foreach \a/\b [count=\s] in {B/-9, A/-16, F/17, E/9, D/4, C/-3}
		{
  			\node[draw, circle, minimum width=1cm] at ({360/\n * (\s - 1)}:\radius) {$\a$};
  			\node at ({360/\n * (\s - 1)}:3.85cm) {$\b$};
  			\draw[>=latex] ({360/\n * (\s - 1)+\margin}:\radius) 
    			arc ({360/\n * (\s - 1)+\margin}:{360/\n * (\s)-\margin}:\radius);
		}

		\node[draw, rectangle, minimum height=1cm] at (-7.0, 0.0) {$data$};
		\draw[dashed, -triangle 45] (-6.95,0.5) arc (135:45:2.8cm);
		\node at (-5.1, 1.8) {$hash: -10$};
		\draw[-triangle 45] (-2.5, 0.0) arc (135:45:3.55cm);
		\draw[-triangle 45] (-2.5, 0.0) arc (90:32:4.5cm);
		\draw[-triangle 45] (-2.5, 0.0) arc (90:-30:1.45cm);
	\end{tikzpicture}

	\caption{Schematyczna ilustracja dystrybucji danych w~bazie danych Apache Cassandra.}
	\label{fig:data_distribution}
\end{figure}

\begin{enumerate}
	\item Każdemu z~węzłów $\{A, B, C, D, E, F\}$ przypisywany jest token, który zawiera się w~zakresie wartości przyjmowanych przez funkcję skrótu. Strategię wybóru tokenu można konfigurować. Przykładową strategią jest wybór losowy. W~omawianym przykładzie węzłom zostały przypisane tokeny o~wartościach $\{-16, -9, -3, 4, 9, 17\}$.
	\item Użytkownik bazy danych przesyła żądanie do dowolnego węzła, który pełni funkcję koordynatora dla danej operacji. Koordynator nadzoruje wpisanie danych do odpowiednich węzłów. W~omawianym przykładzie rolę koordynatora pełni węzeł~$E$.
	\item \label{en:master_node} Każdy węzeł przechowuje dane, których funkcja skrótu zawiera się w~przedziale $(token_{n-1}, token_{n}]$, gdzie $n$ to numer kolejny węzła rosnący zgodnie z~ruchem wskazówek zegara. W~przykładzie węzeł~$C$ przechowuje wiersze o~wartościach funkcji skrótu z~przedziału $(-9, -3]$, natomiast węzeł~$D$ z~przedziału $(-3, 4]$. Wartości funkcji obliczane są cyklicznie, stąd węzeł~$A$ przechowuje wiersze o~skrócie z~przedziału $(-\infty, -16] \cup (17, \infty)$. W~przykładzie wiersz o~kluczu z~funkcją skrótu wartości $-10$ zostanie utrwalony na węźle~$B$.
	\item Dane replikowane są na $n$ węzłach, gdzie $n$ to wartość konfigurowalnego współczynnika replikacji. Poza węzłem macierzystym (wyznaczanym w~punkcie \ref{en:master_node} algorytmu) dane są replikowane na $n - 1$ kolejnych (zgodnie z~ruchem wskazówek zegara) węzłach. W~omawianym przykładzie dane zostaną zreplikowane na węzłach~$C$~i~$D$.
\end{enumerate}
