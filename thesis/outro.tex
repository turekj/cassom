% -*- root: main.tex -*-

\chapter{Podsumowanie}

Tematem pracy było zbadanie możliwości przystosowania istniejących rozwiązań mapowania obiektowo-relacyjnego do bazy danych Apache Cassandra. Celem było zachowanie wysokiej wydajności oraz możliwości stosowania wzorców projektowania, tak aby szybkość rozwiązania mogła być dalej optymalizowana. Już na wstępie pracy okazało się, że taka definicja celów okazała się wzajemnie sprzeczna. Przeprowadzone badania wykazały, że odwzorowanie relacyjnego modelu danych w~Apache Cassandrze jest możliwe, jednakże niweluje to większość zalet tego systemu bazodanowego. Uzyskane wyniki wydajności operowania na takim schemacie były porównywalne z~relacyjnym systemem baz danych. 