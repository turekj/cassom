% -*- root: main.tex -*-

\chapter{Podsumowanie}

Tematem pracy było zbadanie możliwości przystosowania istniejących rozwiązań mapowania obiektowo-relacyjnego do bazy danych Apache Cassandra. Celem było zachowanie wysokiej wydajności oraz możliwości stosowania wzorców projektowania, tak aby szybkość mogła być dowolnie optymalizowana. Już na wstępie pracy okazało się, że taka definicja celów jest wzajemnie sprzeczna. Przeprowadzone badania wykazały, że odwzorowanie relacyjnego modelu danych w~Apache Cassandrze jest możliwe, jednakże niweluje większość zalet tego systemu bazodanowego. Uzyskane wyniki wydajności operowania na takim schemacie były porównywalne z~relacyjnym systemem baz danych.

Z~wzajemnie sprzecznych założeń Autor postanowił zrezygnować z~pełnej zgodności z~istniejącymi narzędziami do mapowania obiektowo-relacyjnego. Umożliwiło to znaczne rozszerzenie zakresu pracy. Zamiast prostego odwzorowania dziedziny danych w~świecie obiektowym udało się zaproponować i~zrealizować kompleksowy mechanizm wspomagający modelowanie, a~także zarządzanie danymi. Rozwiązanie realizuje postawione cele zachowania wydajności i~wspomagania wykorzystywania wzorców projektowych. Dodatkowo, przynajmniej częściowo, spełnia postulaty zgodności z~mapowaniem obiektowo-relacyjnym dla platformy Django. Integracja z~istniejącymi aplikacjami, poza drobnymi modyfikacjami kodu źródłowego, przebiega bezproblemowo.

Praca dowiodła skuteczności podejścia ,,najpierw kod źródłowy'' w~odniesieniu do modelowania dziedziny i~obsługi danych w~Apache Cassandra. Dzięki niemu tworzenie wydajnego schematu jest dużo prostsze. Użytkowanie wysokopoziomowych wzorców modelowania nie wymaga zagłębiania się w~szczegóły implementacyjne. 

