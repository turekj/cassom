% -*- root: main.tex -*-

\chapter{Podsumowanie}

Tematem pracy było zbadanie możliwości przystosowania istniejących rozwiązań mapowania obiektowo-relacyjnego do bazy danych Apache Cassandra. Celem było zachowanie wysokiej wydajności oraz możliwości stosowania wzorców projektowania, tak aby umożliwić łatwą optymalizację. Już na wstępie pracy okazało się, że taka definicja celów jest wzajemnie sprzeczna. Przeprowadzone badania wykazały, że odwzorowanie relacyjnego modelu danych w~Apache Cassandrze jest możliwe, jednakże niweluje większość zalet tego silnika bazodanowego. Uzyskane wyniki wydajności operowania na takim schemacie były porównywalne z~relacyjnymi systemami baz danych.

Autor postanowił zrezygnować z~jednego spośród wzajemnie sprzecznych założeń i~odrzucił pełną zgodność z~istniejącymi narzędziami do mapowania obiektowo-relacyjnego. Umożliwiło to znaczne rozszerzenie zakresu pracy. Zamiast prostego odwzorowania dziedziny danych w~świecie obiektowym udało się zaproponować i~zrealizować kompleksowy mechanizm wspomagający modelowanie, a~także zarządzanie danymi. Rozwiązanie realizuje postawione cele zachowania wydajności i~wspomagania wykorzystywania wzorców projektowych. Dodatkowo, przynajmniej częściowo, spełnia postulaty zgodności z~mapowaniem obiektowo-relacyjnym dla platformy Django. Integracja z~istniejącymi aplikacjami, poza drobnymi modyfikacjami kodu źródłowego, przebiega bezproblemowo.

Praca dowiodła skuteczność podejścia ,,najpierw kod źródłowy'' w~odniesieniu do modelowania dziedziny i~obsługi danych w~Apache Cassandra. Dzięki niemu proces tworzenia i~użytkownia wydajnego schematu jest dużo prostszy. Użytkowanie wysokopoziomowych wzorców modelowania nie wymaga zagłębiania się w~szczegóły implementacyjne, co ułatwia proces uczenia się. Autor ma nadzieję, że przeprowadzone przez niego badania przyczynią się do rozwoju prac nad modelowaniem dziedziny danych w~Apache Cassandrze. W~przyszłości rozwiązania wykorzystujące zaprezentowane w~pracy pomysły mogą stać się standardem komunikacji dla tej bazy danych.

Praca może być dalej rozwijana w~kierunku rozszerzania mechanizmu o~kolejne wzorce modelowania. Ponadto implementacja może być dalej optymalizowana pod względem wydajnościowym. Przede wszystkim chodzi o~usprawnienie procesów eksportowania i~importowania danych z/do Apache Cassandry, a~także migracji. Operacje te są bowiem najbardziej czasochłonne spośród wszystkich zaprezentowanych w~pracy.

Autor pragnie podziękować Michałowi Aniserowiczowi za inspirację do stworzenia mechanizmów migracyjnych, a~także zgodę na wykorzystanie danych osobowych na potrzeby przykładu~\ref{tab:secondary_index_example_users_table}. Bez niego nie byłoby możliwe ukończenie badań w~pełnym zakresie zaprezentowanym w~pracy.
